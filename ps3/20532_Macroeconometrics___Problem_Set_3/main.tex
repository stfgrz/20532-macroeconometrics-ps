\documentclass[dvipsnames,11pt]{article}

%Hey, if you're using this preamble it means that it was probably written by Stefano Graziosi (me). If you see something that doesn't make sense, feel free to email me at stefano.graziosi@studbocconi.it
%p.s. in case it's not already evident from the preamble, I'm not a professional LaTeX user, so I'm sure there are better ways to do things. I'm just trying to make it work.

% =============================================================================
%  Preamble maintained by Stefano Graziosi
%  If anything looks odd, ping: stefano.graziosi@studbocconi.it
%  LAST UPDATE: 01-11-2025
% =============================================================================

% I don't own copyright on anything; credits to original authors.

% =============================================================================
%  CORE PACKAGES (general utilities, fonts/encoding, colors, math, layout)
% =============================================================================

% Headers/footers
\usepackage{fancyhdr}

% Encoding & fonts
\usepackage[T1]{fontenc}
\usepackage{lmodern,mathrsfs}

\usepackage[dvipsnames]{xcolor}                     % Colors (named colors like MidnightBlue, Cerulean, etc.)
\usepackage[many]{tcolorbox}                        % Boxed environments
\usepackage{graphicx}                               % Graphics
\usepackage{bookmark}                               % (Optional, improves bookmark handling)
\usepackage{titlesec}                               % Section titles formatting
\usepackage{enumitem}                               % For fancy enumerates
\usepackage{listings}                               % For the coding environment

% Hyperlinks (kept in original order to preserve behavior)
\usepackage{hyperref}
\hypersetup{
  pdftitle={\jobname},
  pdfauthor={Stefano Graziosi},
  colorlinks=true,
  linkcolor=MidnightBlue,
  citecolor=ForestGreen,
  urlcolor=sgpurple
}

% Math
\usepackage{amsmath, amssymb, amsthm}
\usepackage{mathtools, amsfonts, bm}
\usepackage{thmtools}

% Line spacing & multicolumn text
\usepackage{setspace,multicol}

% Title image & PDF inclusion
\usepackage{titlepic}
\usepackage{pdfpages}

% Fun/extras
\usepackage{halloweenmath}
\usepackage{kantlipsum}

% =============================================================================
%  TABLES
% =============================================================================
\usepackage{array}
\usepackage{tabularx}
\usepackage{booktabs}
\usepackage{threeparttable}
\usepackage{siunitx}
\sisetup{
    round-mode=places,
    round-precision=3,
    table-number-alignment=center
}

\usepackage[nameinlink,noabbrev]{cleveref}

% =============================================================================
%  FIGURES & SUBFIGURES
% =============================================================================
% (graphicx loaded above intentionally to keep original load order)
\usepackage{wrapfig}
\usepackage{float}
\usepackage{caption}
\captionsetup{font=small, labelfont=bf, labelsep=period}
\usepackage{subcaption}

% =============================================================================
%  DIAGRAMS AND PLOTS
% =============================================================================

\usepackage[all]{xy}
\usepackage{tikz}
\usetikzlibrary{calc,arrows.meta,positioning}
% (Optional speed-up for heavy docs; requires -shell-escape)
% \usetikzlibrary{external}
% \tikzexternalize[prefix=tikzcache/]

% =============================================================================
%  Equations
% =============================================================================

\numberwithin{equation}{section} % or \counterwithin{equation}{section}

% =============================================================================
%  PAGE GEOMETRY
% =============================================================================
\usepackage[a4paper,margin=1in]{geometry}
% \usepackage[margin=1in]{geometry} % (alt)

% =============================================================================
%  BIBLIOGRAPHY
% =============================================================================
\usepackage[backend=biber,
            style=numeric,
            sorting=nyt]{biblatex}
\addbibresource{refs.bib}
\usepackage{csquotes}               % smarter quotes & works great with biblatex

% =============================================================================
%  CHAPTER/SECTION FORMATTING & NUMBERING
% =============================================================================

% Keep pure number for counters/refnames
\renewcommand\thesection{\arabic{section}}

% Pretty-print the section heading label
\titleformat{\section}{\normalfont\Large\bfseries}{Question \thesection:}{0.5em}{}

% Figure/Table numbering by section
\usepackage{chngcntr}
\counterwithin{figure}{section}
\counterwithin{table}{section}
\renewcommand{\thefigure}{\thesection-\arabic{figure}}
\renewcommand{\thetable}{\thesection-\arabic{table}}

% =============================================================================
%  COLORS (custom palette)
% =============================================================================

\definecolor{sgblue}{RGB}{0,169,211}
\definecolor{sggreen}{RGB}{0,164,0}
\definecolor{sgpurple}{RGB}{99,0,165}
\definecolor{sgyellow}{RGB}{255,211,0}
\definecolor{sgorange}{RGB}{255,127,20}
\definecolor{sbblue}{RGB}{219,248,254}
\definecolor{sbgreen}{RGB}{223,255,218}
\definecolor{sbpurple}{RGB}{241,220,255}

\definecolor{codegreen}{rgb}{0,0.6,0}
\definecolor{codegray}{rgb}{0.5,0.5,0.5}
\definecolor{codepurple}{rgb}{0.58,0,0.82}
\definecolor{backcolour}{rgb}{0.95,0.95,0.92}

% =============================================================================
%  GENERIC BOX ENVIRONMENT
% =============================================================================

% Standard LaTeX box with color accents
\newtcolorbox{mybox}[3][]{
  colframe = #2!25,
  colback  = #2!10,
  coltitle = #2!20!black,
  title    = {#3},
  #1,
}

% =============================================================================
%  BASIC PROBLEM/SOLUTION ENVIRONMENTS
% =============================================================================

% "Problem" environment
\newtheorem{problem}{Problem}

% Personalized "Solution" environment
\newenvironment{solution}[1][\it{\textcolor{MidnightBlue}{Solution}}]{\textbf{#1. } }{\textcolor{MidnightBlue}{$\square$}}

% =============================================================================
%  THEOREM STYLES & BOXED PRESENTATION
% =============================================================================

% A helper length to fine-tune the label alignment.
% We measure the width of a single normal-space character, store it in \spacelength,
% and later shift the heading by exactly that width (negative hskip) to line things up.
\newlength{\spacelength}
\settowidth{\spacelength}{\normalfont\ }

% ----------------------------- STYLE: "theorem" ------------------------------
% This defines a thmtools STYLE named "theorem".
% Later, any environment declared as \declaretheorem[style=theorem]{<env>}
% will use the fonts/formatting set below for its HEAD LINE (the bold label area).
\declaretheoremstyle[
  headfont={\bfseries\sffamily\footnotesize},  % font for the heading (e.g., "Theorem 2")
  notefont={\normalfont},                      % font for the optional note (the [...] part)
  bodyfont={\normalfont},                      % font for the body text inside the environment
  headpunct={\relax},                          % punctuation after the heading; \relax = none
  % headformat controls how the label is typeset.
  % \NAME   -> the environment name ("Theorem", "Proposition", etc.)
  % \NUMBER -> the auto number (e.g., "2.3")
  % \NOTE   -> the optional note, provided by \begin{theorem}[<NOTE>]
  % \marginparsep is the standard gap used for margin notes; you reuse it as a spacing constant.
  % \makebox[0pt][r]{...} puts the label in a zero-width box, right-aligned,
  % and then \hskip-\spacelength nudges the following text left by one space width
  % so that the body aligns nicely under the heading.
  headformat={\makebox[0pt][r]{\NAME\ \NUMBER\hspace{\marginparsep}}\hskip-\spacelength{\normalsize\NOTE}},
]{theorem}

% ----------------- tcolorbox skin for any env using name "theorem" -----------
% This does NOT create an environment; it says: "whenever the 'theorem'
% environment is used, wrap its contents with this box styling."
\tcolorboxenvironment{theorem}{
  boxrule=0pt,                                % no outer rectangular frame
  boxsep=0pt,                                 % no internal padding around the content (we set our own left/right)
  colback={White},                            % white background
  enhanced jigsaw,                            % enables advanced tcolorbox features
  borderline west={1pt}{0pt}{ForestGreen},    % a 1pt vertical colored bar on the LEFT
  sharp corners,                              % squared corners (not rounded)
  before skip=10pt,                           % vertical space before the box
  after skip=10pt,                            % vertical space after the box
  left=5pt, right=5pt,                        % inner left/right padding (content inset)
  breakable,                                  % allow the box to split across pages
}

% --------------------------- PROPOSITION setup -------------------------------
% Create "proposition" as a thmtools theorem that USES the "theorem" STYLE above.
\declaretheorem[style=theorem]{proposition}

% And give "proposition" its own box skin (purple left border). Same mechanics as "theorem".
\tcolorboxenvironment{proposition}{
  boxrule=0pt,
  boxsep=0pt,
  colback={White},
  enhanced jigsaw,
  borderline west={1pt}{0pt}{Mulberry},
  sharp corners,
  before skip=10pt,
  after skip=10pt,
  left=5pt,
  right=5pt,
  breakable,
}

% ---------------------------- THEOREM setup ----------------------------------
% Explicitly declare the "theorem" environment (thmtools way) using the "theorem" style.
% Note: the box style for "theorem" is already defined above via \tcolorboxenvironment{theorem}{...}.
\declaretheorem[style=theorem]{theorem}

% ----------------------------- STYLE: "proof" --------------------------------
% amsthm already defines a proof environment, so we first neutralize it:
\let\proof\relax
\let\endproof\relax

% Now we declare a thmtools STYLE named "proof" for the heading of the proof environment.
\declaretheoremstyle[
  headfont={\small\scshape},                  % small caps for "Proof"
  notefont={\normalfont},
  bodyfont={\normalfont},
  headpunct={\relax},
  headformat={\makebox[0pt][r]{\NAME\hspace{\marginparsep}}\hskip-\spacelength{\NOTE}},
]{proof}

% And we hook the "proof" environment to a tcolorbox with a thin black left rule.
\tcolorboxenvironment{proof}{
  boxrule=0pt,
  boxsep=0pt,
  blanker,                                    % removes the typical tcolorbox background to look like plain text
  borderline west={1pt}{0pt}{black},          % thin black bar on the left
  before skip=10pt,
  after skip=10pt,
  left=5pt,
  right=5pt,
  breakable,
}

% Now we REDECLARE the actual "proof" environment using the "proof" STYLE.
% The 'qed=\qedsymbol' option ensures the end-of-proof □ is inserted automatically at the end.
\declaretheorem[style=proof, qed=\qedsymbol]{proof}

% ------------------------------ STYLE: "claim" -------------------------------
% A lighter-looking style used for "Intuition" and similar notes (italic header).
\declaretheoremstyle[
  headfont={\footnotesize\itshape},
  notefont={\normalfont},
  bodyfont={\normalfont},
  headpunct={\relax},
  headformat={\makebox[0pt][r]{\NAME\hspace{\marginparsep}}\hskip-\spacelength{\NOTE}},
]{claim}

% Create an "Intuition" environment that uses the 'claim' style (no special box here).
\declaretheorem[style=claim]{Intuition}

% -------------------- Environments declared the amsthm way -------------------
% Below you switch to amsthm’s API: \theoremstyle{<name>} then \newtheorem{...}.
% NOTE: \theoremstyle selects an amsthm style (e.g., plain/definition/remar k),
% not a thmtools style. You happen to use the same style NAME "theorem",
% but that name is defined via thmtools above—not via amsthm’s \newtheoremstyle.
% In many setups, \theoremstyle{theorem} has no effect unless you also defined an
% amsthm style called "theorem". Your boxes still appear because
% \tcolorboxenvironment{<envname>}{...} wraps by ENVIRONMENT NAME.

\theoremstyle{theorem}             % (May be a no-op unless an amsthm style "theorem" exists.)
\newtheorem{ques}{Question}        % Unboxed label styling depends on current amsthm style.

% Definition with its own box skin (Cerulean left rule). The header font/styling here
% comes from the current amsthm \theoremstyle (see the note above).
\theoremstyle{theorem}
\newtheorem{definition}{Definition}
\tcolorboxenvironment{definition}{
  boxrule=0pt,
  boxsep=0pt,
  colback={White},
  enhanced jigsaw,
  borderline west={1pt}{0pt}{Cerulean},
  sharp corners,
  before skip=10pt,
  after skip=10pt,
  left=5pt,
  right=5pt,
  breakable,
}

% Lemma with a Rhodamine bar
\theoremstyle{theorem}
\newtheorem{lemma}{Lemma}
\tcolorboxenvironment{lemma}{
  boxrule=0pt,
  boxsep=0pt,
  blanker,
  borderline west={1pt}{0pt}{Rhodamine},
  before skip=10pt,
  after skip=10pt,
  sharp corners,
  left=5pt,
  right=5pt,
  breakable,
}

% Remark with a BurntOrange bar
\theoremstyle{theorem}
\newtheorem{remark}{Remark}
\tcolorboxenvironment{remark}{
  boxrule=0pt,
  boxsep=0pt,
  colback={White},
  enhanced jigsaw,
  borderline west={1pt}{0pt}{BurntOrange},
  before skip=10pt,
  after skip=10pt,
  sharp corners,
  left=5pt,
  right=5pt,
  breakable,
}

% Corollary with a WildStrawberry bar
\theoremstyle{theorem}
\newtheorem{corollary}{Corollary}
\tcolorboxenvironment{corollary}{
  boxrule=0pt,
  boxsep=0pt,
  enhanced jigsaw,
  borderline west={1pt}{0pt}{WildStrawberry},
  before skip=10pt,
  after skip=10pt,
  sharp corners,
  left=5pt,
  right=5pt,
  breakable,
}

% Example with a Dandelion bar
\theoremstyle{theorem}
\newtheorem{example}{Example}
\tcolorboxenvironment{example}{
  boxrule=0pt,
  boxsep=0pt,
  blanker,
  borderline west={1pt}{0pt}{Dandelion},
  before skip=10pt,
  after skip=10pt,
  sharp corners,
  left=5pt,
  right=5pt,
  breakable,
}

% Additional helpers using the 'claim' style (amsthm path).
% Again, \theoremstyle{claim} here refers to an amsthm style named "claim".
% You defined a thmtools style named "claim" above, which doesn't automatically
% register with amsthm. If no amsthm style "claim" exists, this line may be a no-op.
\theoremstyle{claim}
\newtheorem{intu}{Intuition}

\theoremstyle{claim}
\newtheorem{solu}{Solution}

% =============================================================================
%  CODE LISTINGS
% =============================================================================

\lstdefinestyle{mystyle}{
  backgroundcolor=\color{backcolour},
  commentstyle=\color{codegreen},
  keywordstyle=\color{magenta},
  numberstyle=\tiny\color{codegray},
  stringstyle=\color{codepurple},
  basicstyle=\ttfamily\footnotesize,
  breakatwhitespace=false,
  breaklines=true,
  captionpos=b,
  keepspaces=true,
  numbers=left,
  numbersep=5pt,
  showspaces=false,
  showstringspaces=false,
  showtabs=false,
  tabsize=2
}

\lstset{style=mystyle}


\setlength{\headheight}{25pt} % Non so perché ma senza questo da un piccolo errore

\pagestyle{fancy}
\fancyhf{} 
\fancyhead[L]{\textsc{\small{20532\\ Macroeconometrics}}} 
\fancyhead[R]{\textsc{\small Stefano Graziosi}} 
\fancyhead[C]{\textbf{Problem set 3}}
\fancyfoot[R]{\thepage} 
\fancyfoot[L]{\small \today} 
\renewcommand{\headrulewidth}{0.4pt} 
\renewcommand{\footrulewidth}{0pt}

\begin{document}

\section{Cholesky identification and monetary policy shocks}

    In the dataset, in the first sheet of \texttt{data\_ps3.xlsx}, you will find the following variables: \texttt{log(\text{real GDP})}, \texttt{log(\text{price deflator})}, and the federal funds rate for the U.S. from 1960 to 2010 at quarterly frequency. Estimate a VAR on those variables, identify the monetary policy shock with a triangular (Cholesky) identification structure with the ordering above, and compute impulse responses and variance decompositions for the monetary policy shock. Compute confidence intervals around the point estimates for the impulse responses using a bootstrap.
    
    \begin{hint}
        A bootstrap is similar to a Monte Carlo, but instead of sampling from a known distribution, you draw from the estimated residuals.
    \end{hint} 
    
    The steps are:
    
    \begin{enumerate}[label=\alph*.]
        \item Estimate the VAR (suppose with a constant) on the $N$ variables in $y_t$ and store the coefficients (with the constants collected in the $(N\times1)$ vector $\hat c$ and the $(N\times N)$ autoregressive coefficients in the matrix $\hat A$). 
        
        If there are $p$ lags, you can express the VAR($p$) as a VAR(1) with the companion form, and the $(T\times N)$ residuals $\hat\varepsilon$ (note that $T$ is the number of estimated residuals, those obtained once one eliminated the lags lost in the estimation).

            \begin{solution}

                We begin by importing the data and setting up the problem

\begin{lstlisting}[language=Matlab]
% Settings
H = 12;     % IRF / FEVD horizon
K = 5000;   % Number of bootstrap replications

% Data Loading
file_ex1 = 'ps3/data/ps3_monetary_shock.csv';
data_ex1 = readtable(file_ex1, 'PreserveVariableNames', true);

Y = [data_ex1.log_gdp, data_ex1.log_p, data_ex1.ffr];
labels = ["log GDP", "log Price Level", "FFR"];
[n_obs, n_vars] = size(Y);
\end{lstlisting}

                Let $y_t\in\mathbb{R}^N$ collect $\{\log(\text{real GDP}), \log(\text{price deflator}), \text{FFR}\}$ in that order. We can now estimate a VAR(p) in reduced form with an intercept as:
                \begin{equation}
                    y_t \;=\; c \;+\; A_1 y_{t-1} \;+\; \cdots \;+\; A_p y_{t-p} \;+\; u_t,\qquad
                    \mathbb{E}[u_t u_t'] \;=\; \Sigma_u.
                \end{equation}
                Equation-by-equation OLS is consistent for $\{c,A_1,\ldots,A_p\}$ and yields the residual covariance $\Sigma_u$. 
                
                For computing impulse responses (and as suggested in the text of the assignment), it is convenient to stack into the VAR(1) companion form,
                \begin{equation}
                    \mathbf{Y}_t \;=\; C \;+\; \mathcal{A}\,\mathbf{Y}{t-1} \;+\; U_t,\quad
                    \mathbf{Y}_t =
                    \begin{bmatrix}
                    \mathbf{y}_t\\
                    \mathbf{y}_{t-1}\\
                    \vdots\\
                    \mathbf{y}_{t-p+1}
                    \end{bmatrix},
                    \qquad
                    \mathcal{A} =
                    \begin{bmatrix}
                    A_1 & A_2 & \cdots & A_{p-1} & A_p\\
                    I_k & \mathbf{0} & \cdots & \mathbf{0} & \mathbf{0}\\
                    \mathbf{0} & I_k & \ddots & \vdots & \vdots\\
                    \vdots & \ddots & \ddots & \mathbf{0} & \mathbf{0}\\
                    \mathbf{0} & \cdots & \mathbf{0} & I_k & \mathbf{0}
                    \end{bmatrix},
                \end{equation}
                which implies a vector MA (VMA) representation underpinning impulse responses.
                
                To identify structural (orthogonal) shocks $\varepsilon_t$, we impose a recursive (triangular) structure via the Cholesky factor of $\Sigma_u$. Specifically, define the $N\times N$ lower-triangular impact matrix $P$ such that
                \begin{equation}
                    u_t \;=\; P\,\varepsilon_t,\qquad \mathbb{E}[\varepsilon_t\varepsilon_t']=I_N,
                \end{equation}
                which delivers exact identification by setting $(N^2-N)/2$ contemporaneous coefficients to zero. With variables ordered \emph{(GDP, price level, FFR)}, the monetary policy shock is last and may contemporaneously move all variables, while output is contemporaneously insulated and the price level reacts contemporaneously to output but not to policy. This is the standard Sims-style recursive identification (a minimal set of assumptions) and fixes a unique mapping from reduced-form to structural shocks. 
                
                The implied structural IRFs at horizon $h$ are
                \begin{equation}
                    \Psi_h \;=\; J\,\mathcal{A}^h J'\,P,\qquad
                    J=\begin{bmatrix} I_N & 0\end{bmatrix},
                \end{equation}
                and are read columnwise by shock and rowwise by variable.

                Moving on to the implementation, we choose the VAR lag length by estimating $\text{VAR}(p)$ models for $p=1,\dots,8$ and computing three systemwide information criteria, namely AIC, BIC and HQC, for each $p$. For a given $p$, let $U_p$ be the $T\times N$ matrix of OLS residuals and $S_p = (U_p'U_p)/T$ the maximum-likelihood estimate of the residual covariance. The criterion takes the form
                \begin{equation}
                    \mathrm{IC}(p) = \log\!\det S_p \;+\; \frac{c_T\,k(p)}{T},
                    \quad\text{with}\quad
                    k(p)=N^2p+N,
                \end{equation}
                where $c_T=2$ (AIC), $c_T=\log T$ (BIC), and $c_T=2\log\log T$ (HQC).

\begin{lstlisting}[language=Matlab]
maxp = 8; 
IC  = nan(maxp,3);   % [AIC, BIC, HQC]
models = cell(maxp,1);

for pp = 1:maxp
    m = estimateVAR(Y, pp);
    T = m.n_eff;                     % effective sample = n_obs - pp
    k = m.n_vars^2 * pp + m.n_vars;  % total parameters (system-wide)

    % ML covariance for information criteria
    S = (m.U' * m.U) / T; 
    S = (S + S')/2;                  % enforce symmetry

    % robust log |S|
    [R,pflag] = chol(S); 
    if pflag ~= 0
        % gentle jitter if needed
        [V,D] = eig(S);
        D = max(D, 1e-12*eye(size(D)));
        S  = V*D*V';
        R  = chol(S);
    end
    ll = 2*sum(log(diag(R)));        % = logdet(S)

    % ICs per standard VAR formulas
    IC(pp,1) = ll + (2*k)/T;           % AIC
    IC(pp,2) = ll + (log(T)*k)/T;      % BIC
    IC(pp,3) = ll + (2*log(log(T))*k)/T; % HQC

    models{pp} = m;
end

[~,pAIC] = min(IC(:,1));
[~,pBIC] = min(IC(:,2));
[~,pHQC] = min(IC(:,3));

fprintf('Selected lag orders:\n AIC: %d\n BIC: %d\n HQC: %d\n', pAIC, pBIC, pHQC);

p = pHQC;
\end{lstlisting}

                    From a theorical perspective BIC is consistent but can underfit in short samples; AIC often overfits; HQC is a common compromise. We therefore adopt the HQC-selected $p$ (here $p=4$).

                    For what it concerns the estimation of the VAR, given the repetitiveness of the task, we opt for a function-based approach by defining the four following helper functions, which we will use to estimate and store all of the required variables, namely (i) the VAR parameters, (ii) the impact matrix, (iii) the Impulse Response Functions (IRF), and (iv) the Forecast Error Variance Decompositions.

                \begin{enumerate}[label=\roman*.]
                    \item \texttt{estimateVAR(Y, p)}

                        Builds and estimates a VAR($p$) with an intercept by OLS, then assembles the stacked coefficient matrix $A=[A_1\ \cdots\ A_p]$ and the companion matrix $\mathcal{A}$. We store: coefficients (\texttt{B}), residuals (\texttt{U}), residual covariance (\texttt{SigmaU}), intercept vector (\texttt{c}), stacked $A$, and $\mathcal{A}$. Note that

                        \begin{itemize}
                            \item We trim the first $p$ observations consistently in $X$ and $Y$.
                            
                            \item \texttt{SigmaU} uses $df=T-k$ (unbiased); this differs from the ML scaling ($T$) but does not affect identification or IRFs.
                            
                            \item The orientation \(\texttt{A\_k = B(rows\_k,:).'}\) is crucial so that $y_t = c + \sum_{k=1}^p A_k y_{t-k} + u_t$ holds with column-vector dynamics.
                            
                            \item The companion matrix later lets us compute $\Phi_h = J\mathcal{A}^h J'$ efficiently.
                        \end{itemize}

\begin{lstlisting}[language=Matlab]
function model = estimateVAR(Y, p)
    % Estimates a VAR(p) model with an intercept.
    % Y: (n_obs x n_vars) data matrix
    % p: lag order
    [n_obs, n_vars] = size(Y);
    
    X_lags = mlag(Y, p);
    X = [ones(n_obs, 1), X_lags];
    
    Y_trim = Y(p+1:end, :);
    X_trim = X(p+1:end, :);
    
    model.B = X_trim \ Y_trim; % OLS coefficients
    model.U = Y_trim - X_trim * model.B; % Residuals
    
    model.n_eff = size(Y_trim, 1);
    model.k_reg = size(model.B, 1);
    df = model.n_eff - model.k_reg;
    
    model.SigmaU = (model.U' * model.U) / df; % Residual covariance matrix
    
    % Store coefficients in a structured way
    model.c = model.B(1, :).'; % (n_vars x 1)

    % Build A = [A1 A2 ... Ap] with correct orientation (column-vector dynamics)
    A = zeros(n_vars, n_vars*p);
    for k = 1:p
        rows_k = (2 + (k-1)*n_vars) : (1 + k*n_vars); % rows in B for lag k
        A_k = model.B(rows_k, :).';                    % transpose is crucial
        A(:, (n_vars*(k-1)+1):(n_vars*k)) = A_k;
    end
    model.A = A;

    % Companion matrix
    model.A_comp = [A;
                    eye(n_vars*(p-1)), zeros(n_vars*(p-1), n_vars)];

    % Store other useful info
    model.p = p;
    model.n_vars = n_vars;
end          
\end{lstlisting}          

                    \item \texttt{identifyCholesky(VAR\_orig.SigmaU)}

                        Returns the lower-triangular impact matrix $P$ such that $u_t=P\varepsilon_t$ and $\mathbb{E}[\varepsilon_t\varepsilon_t']=I$. Under the chosen ordering (GDP $\rightarrow$ price $\rightarrow$ FFR), the policy shock can move all variables contemporaneously, while output is insulated on impact. Note that:
                        \begin{itemize}
                            \item We request the \emph{lower} Cholesky factor to match the standard recursive (triangular) scheme.
                            
                            \item A gentle SPD “nudge” handles near-singular covariances (common with short samples/high $p$). 
                            
                            \item $P$ normalizes shocks to unit variance; sign conventions are set by the positive diagonal of $P$ (flip a column if a different sign convention is desired for the policy shock).
                        \end{itemize}

\begin{lstlisting}[language=Matlab]
function P = identifyCholesky(SigmaU)
    % Computes the Cholesky factor of the residual covariance matrix.
    try
        P = chol(SigmaU, 'lower');
    catch
        % Fallback for matrices that are not perfectly positive definite
        warning('SigmaU is not positive definite. Nudging to nearest SPD matrix.');
        Sig = (SigmaU + SigmaU') / 2;
        [V, D] = eig(Sig);
        D = max(D, 1e-12 * eye(size(SigmaU,1)));
        P = chol(V * D * V', 'lower');
    end
end       
\end{lstlisting}    

                    \item \texttt{computeIRF(VAR\_orig, P\_orig, H)}

                        Computes $\Psi_h = \Phi_h P$ for $h=0,\dots,H$, where $\Phi_h = J\mathcal{A}^h J'$ are the reduced-form MA coefficients from the companion powers. Output is an array \((n\_\text{vars}\times n\_\text{vars}\times (H{+}1))\): rows = responses, columns = shocks, pages = horizons. Note that:
                        \begin{itemize}
                            \item $h=0$ gives impact responses. 
                            
                            \item Because $P$ orthonormalizes shocks, columns of $\Psi_h$ correspond to 1-s.d. structural shocks. 
                        \end{itemize}

\begin{lstlisting}[language=Matlab]
function IRF = computeIRF(VAR_model, P, H)
    % Computes structural impulse responses.
    % VAR_model: A struct from estimateVAR
    % P: (n_vars x n_vars) structural impact matrix
    % H: Horizon for IRFs
    
    [n_comp, ~] = size(VAR_model.A_comp);
    n = VAR_model.n_vars;
    
    IRF = zeros(n, n, H+1);
    J = [eye(n), zeros(n, n_comp - n)]; % Selection matrix
    
    A_pow = eye(n_comp);
    for h = 0:H
        Phi_h = J * A_pow * J'; % Reduced-form MA coefficient
        IRF(:,:,h+1) = Phi_h * P; % Structural IRF
        A_pow = A_pow * VAR_model.A_comp;
    end
end       
\end{lstlisting}   

                    \item \texttt{computeFEVD(IRF\_orig)}

                        Given structural IRFs $\Psi_h$, the $n$-step forecast error of variable $i$ is
                        $\sum_{s=0}^{n}\sum_{j}\Psi_{s,ij}\,\varepsilon_{j,t+s}$, so with orthonormal shocks the FEVD share for shock $j$ equals the cumulated squared contribution of column $j$ divided by the cumulated total. The code implements
                        \begin{equation}
                            \text{FEVD}_{i j}(h)=\frac{\sum_{s=0}^{h} \Psi_{s,ij}^2}{\sum_{j'}\sum_{s=0}^{h} \Psi_{s,ij'}^2}, \quad h=0,\dots,H.
                        \end{equation}
                        Note that:
                        \begin{itemize}
                            \item Rows sum to one (up to round-off). 
                            
                            \item Order sensitivity is greatest at short horizons under Cholesky; this is expected and typically fades with $h$.
                        \end{itemize}

\begin{lstlisting}[language=Matlab]
function FEVD = computeFEVD(IRF)
    % Computes Forecast Error Variance Decomposition from structural IRFs.
    % IRF: (n_vars x n_vars x H+1) array of impulse responses
    
    [n, ~, H1] = size(IRF);
    H = H1 - 1;
    
    FEVD = zeros(n, n, H1);
    
    % Cumulative sum of squared IRFs over horizons
    irf_sq_cumsum = cumsum(IRF.^2, 3);
    
    for h = 1:H1
        total_variance = sum(irf_sq_cumsum(:,:,h), 2); % Sum over shocks for each variable
        if any(total_variance > 0)
            FEVD(:,:,h) = irf_sq_cumsum(:,:,h) ./ total_variance;
        end
    end
end         
\end{lstlisting}   

                \end{enumerate}

                At this point, it is sufficient to call the four functions using the parameters we set up in the settings, together with the ones we obtained throughout this point:

\begin{lstlisting}[language=Matlab]
VAR_orig = estimateVAR(Y, p);

P_orig = identifyCholesky(VAR_orig.SigmaU);

IRF_orig = computeIRF(VAR_orig, P_orig, H);

FEVD_orig = computeFEVD(IRF_orig);
\end{lstlisting}  
        
            \end{solution}  
        
        \item Sample with replacement from the estimated residuals so to form a new series of residuals $\tilde\varepsilon$ of dim $(T\times N)$. 

        \begin{hint}
            Sample \emph{with} replacement.
        \end{hint}

        \begin{hint}
            Sample one entire row (of dim $(1\times N)$) of the matrix $\tilde\varepsilon$.
        \end{hint}
        
        One way of doing it is generate $T$ random integers from $1$ to $T$ (for example, call the random draw of integers \texttt{PER}) and then set $\tilde\varepsilon = \hat\varepsilon[\texttt{PER},:]$; don’t use the command \texttt{permute}.

            \begin{solution}

                To place inference bands on IRFs without strong parametric assumptions, we use a residual-based bootstrap. As recommended, we draw with replacement from the estimated reduced-form residuals \emph{rowwise} to preserve their cross-equation correlation structure, i.e.,
                \begin{equation}
                    \tilde U \;=\; \bigl\{\,u_{t^\star}: t^\star \in \{1,\ldots,T\}\text{ drawn i.i.d. with replacement}\,\bigr\}.
                \end{equation}
                Sampling entire residual vectors $u_t'\in\mathbb{R}^N$ keeps the contemporaneous covariance of innovations intact; this is essential because the Cholesky identification relies on the (co)variance structure of the residuals. Enders recommends resampling the paired residuals $\{e_{1t},e_{2t}\}$ together for exactly this reason. 

                \begin{warning}[Why we're not using permute]
                    Using \texttt{permute} on a single dimension would break the across-equation dependence and misrepresent the joint distribution.
                \end{warning}
                
                We compute percentile bands for each IRF by taking the empirical 2.5\% and 97.5\% quantiles across bootstrap replications to form 95\% confidence intervals.

\begin{lstlisting}[language=Matlab]
fprintf('Bootstrapping IRFs for Exercise 1: K=%d, VAR(%d), H=%d\n', K, p, H);

IRF_draws = zeros(n_vars, n_vars, H+1, K);
FEVD_draws = zeros(n_vars, n_vars, H+1, K);

Y_initial = Y(1:p, :); % Initial conditions for simulation
U_centered = VAR_orig.U - mean(VAR_orig.U, 1);
\end{lstlisting}
        
            \end{solution}  

      
        \item Use the newly generated residuals and the estimated coefficients to construct new series:
            \[
            \tilde y_t \;=\; \hat c \;+\; \hat A\,\tilde y_{t-1} \;+\; \tilde\varepsilon_t.
            \]
            The starting values are the first values of $y_t$ (in the case of a VAR(1), just $y_1$).

            \begin{solution}
            
                Using the estimated coefficients and the bootstrapped residuals, we generate artificial data from the fitted VAR:
                    \begin{equation}
                        \tilde y_t \;=\; \hat c \;+\; \hat A_1 \tilde y_{t-1} \;+\; \cdots \;+\; \hat A_p \tilde y_{t-p} \;+\; \tilde u_t,
                    \end{equation}
                    starting from the observed initial conditions $\{y_1,\ldots,y_p\}$. This produces a bootstrap sample $\{\tilde y_t\}_{t=1}^T$ consistent with the estimated dynamics and the empirical innovation distribution. The VMA logic then yields IRFs from the companion powers as in part a.
                    
            \end{solution}
      
        \item Estimate a VAR on the new series $\tilde y_t$; identify shocks and compute im\-pulse responses and variance decompositions. Store the impulse responses and variance decompositions. 
      
        \item Repeat steps b. to d. \(K\) times (e.g., \(K=1000\)).

            \begin{solution}

                We implement steps c. to e. in a single parallel computing loop, so that the implementation looks like the following

\begin{lstlisting}[language=Matlab]
tic
parfor k_boot = 1:K
    % (b) Resample residuals
    U_boot = U_centered(randi(VAR_orig.n_eff, VAR_orig.n_eff, 1), :);
    
    % (c) Simulate new data series
    Y_sim = simulateVAR(VAR_orig.B, Y_initial, U_boot);

    % (d) Re-estimate VAR on simulated data, identify, and compute IRFs/FEVDs
    VAR_boot = estimateVAR(Y_sim, p);
    P_boot   = identifyCholesky(VAR_boot.SigmaU);
    IRF_draws(:,:,:,k_boot)  = computeIRF(VAR_boot, P_boot, H);
    FEVD_draws(:,:,:,k_boot) = computeFEVD(IRF_draws(:,:,:,k_boot));
end
toc
\end{lstlisting}


                \textbf{FEVD definition and interpretation.} 
                    
                    At horizon $n$, the $j$th variable’s $n$-step forecast error can be written as a sum of current and lagged structural shocks with weights given by the VMA coefficients; the FEVD is the share of its forecast error variance attributable to each shock. In a recursive Cholesky scheme, the one-step FEVD of variable 1 is entirely due to shock 1; variable 2’s is split between shocks 1–2; variable 3’s between shocks 1–3, etc. Sensitivity to ordering is greatest at short horizon, and hence reporting FEVDs across several horizons is informative. 

                    Despite not being part of the assignment, we opted for a graphical representation of the evolution of the FEVD for the three variables that we are studying, implemented through the code below:

\begin{lstlisting}[language=Matlab]
shock_names = labels;                 % same names for shocks and vars
t = 0:H;

% Normalize each bootstrap draw so shares sum to 1 at each (var, horizon)
FEVD_sum   = sum(FEVD_draws, 2);                     % [n x 1 x H+1 x K]
FEVD_norm  = bsxfun(@rdivide, FEVD_draws, FEVD_sum); % [n x n x H+1 x K]

% Percentiles across draws (properly normalized)
FEVD_pct   = prctile(FEVD_norm, [2.5, 50, 97.5], 4);
FEVD_med   = squeeze(FEVD_pct(:,:,:,2));  % median shares [n x n x H+1]

% Nice stacked area per variable
for i_var = 1:n_vars
    fh = figure('Position',[80 80 900 520]); hold on;
    A = squeeze(FEVD_med(i_var,:,:))';  % [H+1 x n] for area()
    area(t, A, 'LineStyle','none'); grid on;
    ylim([0 1]); yticks(0:0.2:1); yticklabels(compose('%d%%',0:20:100));
    xlim([0 H]);
    xlabel('Horizon (Quarters)');
    ylabel('Share of forecast error variance');
    title(sprintf('FEVD of %s (median across %d bootstraps)', labels(i_var), size(FEVD_draws,4)));
    legend(shock_names, 'Location','eastoutside');
    set(gca,'Layer','top'); % grid above areas

    % Export
    fname = sprintf('ex1_fevd_stack_%s.pdf', strrep(lower(labels(i_var)),' ','_'));
    exportFig(fh, fname); close(fh);
end
\end{lstlisting}
                
                \textbf{On the monetary ordering used here.} 
                
                    With $\{\log\text{GDP}, \log P, \text{FFR}\}$, the recursive structure implies the policy (FFR) shock is ``causally last'' contemporaneously—a standard assumption in monetary VARs that lets policy react within-quarter to real activity and prices, while real quantities do not jump contemporaneously to policy surprises. Enders emphasizes that such recursive identification is a minimal and transparent way to achieve exact identification; nonetheless, researchers should acknowledge that alternative plausible orderings can alter short-horizon IRFs/FEVDs when reduced-form residuals are highly correlated. 


                        \begin{figure}[H]
                            \centering
                                \begin{subfigure}[b]{0.7\textwidth}
                                    \centering
                                    \includegraphics[width=\textwidth, trim=0cm 0cm 0cm 0.55cm, clip]{output/ex1_fevd_stack_ffr.pdf}
                                    \caption{FEVD of FFR}
                                    \label{fig:ex1_fevd_stack_ffr}
                                \end{subfigure}
                              \hfill
                                \begin{subfigure}[b]{0.7\textwidth}
                                    \centering
                                    \includegraphics[width=\textwidth, trim=0cm 0cm 0cm 0.55cm, clip]{output/ex1_fevd_stack_log_gdp.pdf}
                                    \caption{FEVD of log GPD}
                                    \label{fig:ex1_fevd_stack_log_gdp}
                                \end{subfigure}
                              \hfill
                                \begin{subfigure}[b]{0.7\textwidth}
                                    \centering
                                    \includegraphics[width=\textwidth, trim=0cm 0cm 0cm 0.55cm, clip]{output/ex1_fevd_stack_log_price_level.pdf}
                                    \caption{FEVD of log Price Level}
                                    \label{fig:ex1_fevd_stack_log_price_level}
                                \end{subfigure}
                            \caption{Visual representations of the evolution of the FEVD (median across 5000 bootstraps).}
                            \label{fig:ex_2c}
                        \end{figure}
        
            \end{solution}  
      
        \item At the end you have a set of 1000 impulse responses. Plot the 2.5\% and 97.5\% percentile (command \texttt{prctile}) of that empirical distribution. This is your 95\% confidence interval.

            \begin{solution}

                We construct point estimates, medians across draws, and 95\% percentile bands (2.5–97.5). Figures plot the solid point estimate, a dashed bootstrap median, and a shaded 95\% band over $H=12$ quarters for each structural shock. This mirrors standard innovation-accounting displays and makes short- vs long-horizon uncertainty visually clear.

\begin{lstlisting}[language=Matlab]
IRF_bands = prctile(IRF_draws, [2.5, 50, 97.5], 4);

for j_shock = 1:n_vars
    fh = figure('Position',[90 90 880 760]);
    tlo = tiledlayout(n_vars, 1, 'Padding', 'compact', 'TileSpacing', 'compact');
    
    for i_var = 1:n_vars
        nexttile; hold on; grid on;
        
        % Extract IRFs for the current plot
        irf_point = squeeze(IRF_orig(i_var, j_shock, :));
        irf_lo = squeeze(IRF_bands(i_var, j_shock, :, 1));
        irf_med = squeeze(IRF_bands(i_var, j_shock, :, 2));
        irf_hi = squeeze(IRF_bands(i_var, j_shock, :, 3));
       
        % Plot 95% confidence band
        fill([0:H, H:-1:0], [irf_lo', fliplr(irf_hi')], ...
            [0.85 0.9 1.0], 'EdgeColor', 'none', 'FaceAlpha', 0.8, 'DisplayName', '95% Band');
        
        % Plot point estimate and bootstrap median
        plot(0:H, irf_point, '-', 'LineWidth', 1.8, 'DisplayName', 'Estimate');
        plot(0:H, irf_med, '--', 'LineWidth', 1.2, 'DisplayName', 'Bootstrap Median');
        
        yline(0, 'k:', 'HandleVisibility', 'off');
        ylabel(labels(i_var));
      
        if i_var == 1
            title(sprintf('Responses to a "%s" Shock (Cholesky)', labels(j_shock)));
            legend('Location','best','Interpreter','none');
        end
        if i_var == n_vars, xlabel('Horizon (Quarters)'); end
    end
    
    filename = sprintf('ex1_irf_bands_shock_%s.pdf', strrep(lower(labels(j_shock)), ' ', '_'));
    exportFig(fh, filename);
    close(fh);
end
\end{lstlisting}

                \begin{figure}[H]
                    \centering
                    \includegraphics[width=0.75\linewidth]{output/ex1_irf_bands_shock_ffr.pdf}
                    \caption{Visual representation of a response to a monetary (Federal Funds Rate) shock.}
                    \label{fig:ex1_irf_bands_shock_ffr}
                \end{figure}

                \begin{figure}[H]
                    \centering
                    \includegraphics[width=0.75\linewidth]{output/ex1_irf_bands_shock_log_gdp.pdf}
                    \caption{Visual representation of a response to a logGDP (output) shock.}
                    \label{fig:ex1_irf_bands_shock_log_gdp}
                \end{figure}

                \begin{figure}[H]
                    \centering
                    \includegraphics[width=0.75\linewidth]{output/ex1_irf_bands_shock_log_price_level.pdf}
                    \caption{Visual representation of a response to a logPrice (price level) shock.}
                    \label{fig:ex1_irf_bands_shock_log_price_level}
                \end{figure}
        
            \end{solution}  

    \end{enumerate}

\section{Long-run identification}

    Read the paper \emph{``\href{https://www.aeaweb.org/articles?id=10.1257/aer.89.1.249}{Technology, Employment, and the Business Cycle: Do Technology Shocks Explain Aggregate Fluctuations?}''} by Jordi Gal\'{\i} \cite{10.1257/aer.89.1.249}, \emph{American Economic Review}, 89(1), 1999, 249--271.
    
    Using the dataset in the second sheet of \texttt{data\_ps3.xlsx}, replicate Figure 2 in the paper and compute bootstrapped confidence bands.

    \begin{warning}
        Results may slightly differ from those in the paper, as the data are not exactly the same.
    \end{warning}

    \begin{solution}

        Following \cite{10.1257/aer.89.1.249}, I estimate a bivariate VAR in growth rates of labor productivity ($\Delta x_t$) and hours ($\Delta n_t$) and identify two orthogonal shocks: a “technology” shock and a “nontechnology” shock. The key restriction is that only the technology shock can have a permanent effect on the \emph{level} of labor productivity. In VAR terms, this implies that the long-run impact matrix $C(1)$ is lower triangular, i.e. the cumulated effect of the nontechnology shock on $x_t$ is zero. This is the Blanchard–Quah long-run restriction specialized to productivity and hours, and it is exactly the restriction used by Galí to disentangle technology from nontechnology shocks. 

            \paragraph{Mapping to the structural VAR.}

                Let $e_t$ denote VAR residuals and $\varepsilon_t$ the structural shocks. The long-run restriction is imposed by recovering the long-horizon multipliers from the estimated VAR and choosing the contemporaneous mapping $P$ so that $C(1)P$ is lower triangular. This follows the standard derivation linking $e_t$ and $\varepsilon_t$ through short-run impact and long-run multipliers in the BQ framework. 

        \subsection{Estimating the VAR}

                Panel A of Figure~2 in \cite{10.1257/aer.89.1.249} uses a bivariate system for productivity growth and first-differenced hours. I mimic that setup by constructing $Y_t=[\Delta\log(\text{Productivity})_t,\ \Delta\log(\text{Hours})_t]$ and estimating a VAR($p$) with a constant. I set $p=4$ quarterly lags, horizon $H=12$, and compute IRFs for both identified shocks. This matches Galí’s bivariate Figure~2 configuration (first-differenced hours and two-s.e. bands). 

\begin{lstlisting}[language=Matlab]
% Settings
p_gali = 4;
H_gali = 12;
K_gali = 1000;

% Data Prep (Gali 1999, Panel A)
file_ex2 = 'ps3/data/ps3\_technology_shock.csv';
data_ex2 = readtable(file_ex2, 'PreserveVariableNames', true);

% VAR uses growth rates: \Deltalog(Productivity), \Deltalog(Hours)
prod_level = log(data_ex2.y_l);
hours_level = log(data_ex2.hours);

Y_gali = [diff(prod_level), diff(hours_level)];
labels_gali = ["Productivity", "Hours"];
[~, n_vars_gali] = size(Y_gali);

% Estimate VAR on original data and identify
VAR_gali_orig = estimateVAR(Y_gali, p_gali);
\end{lstlisting}

            The call \texttt{P\_gali\_orig = identifyLongRun(VAR\_gali\_orig)} computes the contemporaneous mapping $P$ so that the long-run multiplier matrix for the VAR’s moving-average representation is lower triangular. Economically, this enforces that the nontechnology shock has no permanent effect on productivity, while allowing permanent effects on hours and output (as in Galí’s specification).

\begin{lstlisting}
P_gali_orig = identifyLongRun(VAR_gali_orig);
IRF_gali_orig = computeIRF(VAR_gali_orig, P_gali_orig, H_gali);
\end{lstlisting}

        \subsection{Bootstrapped confidence bands}

            We use a residual (i.i.d.) bootstrap to quantify parameter uncertainty in the IRFs. Concretely: 
            \begin{enumerate}[label=\roman*.]
                \item center and resample the estimated VAR residuals with replacement;
                
                \item simulate pseudo-data from the estimated VAR using those resampled shocks and the observed initial conditions;
                
                \item re-estimate the VAR and re-identify the shocks in each draw;
                
                \item recompute IRFs and take empirical percentiles across draws. This is the standard residual-bootstrap approach for IRF confidence intervals (closely related to the Monte Carlo bands discussed in Enders). 
            \end{enumerate}
            Galí reports two-s.e. bands generated from 500 Monte Carlo draws, while I use $K=1000$ bootstrap replications. Minor differences in band width are expected. 

\begin{lstlisting}[language=Matlab]
% Bootstrap confidence intervals
fprintf('Bootstrapping IRFs for Exercise 2: K=%d, VAR(%d), H=%d\n', K_gali, p_gali, H_gali);

IRF_gali_draws = zeros(n_vars_gali, n_vars_gali, H_gali+1, K_gali);
Y_gali_initial = Y_gali(1:p_gali, :);

U_centered_gali = VAR_gali_orig.U - mean(VAR_gali_orig.U,1);

tic
parfor k_boot = 1:K_gali
    % Resample residuals
    U_boot = U_centered_gali(randi(VAR_gali_orig.n_eff, VAR_gali_orig.n_eff, 1), :);
    
    % Simulate new data series
    Y_sim = simulateVAR(VAR_gali_orig.B, Y_gali_initial, U_boot);
    
    % Re-estimate VAR, identify, and compute IRFs
    VAR_boot = estimateVAR(Y_sim, p_gali);
    P_boot = identifyLongRun(VAR_boot);
    IRF_gali_draws(:,:,:,k_boot) = computeIRF(VAR_boot, P_boot, H_gali);
end
toc
\end{lstlisting}

            Because the VAR is in growth rates, I cumulate the IRFs over the horizon to obtain level responses. I also report the response of (log) output as the identity $y_t = x_t + n_t$, so $\Delta y_t = \Delta x_t + \Delta n_t$; hence the IRF of output growth equals the sum of the productivity and hours IRFs, and its level response is the cumulated sum of that composite. This matches the three-row layout in Galí’s figures (productivity, output, hours). 

\begin{lstlisting}[language=Matlab]
% Transform IRFs to cumulative levels
    % Function to convert growth rate IRFs to cumulative levels for plotting
cumulate_irfs = @(irf_draws) cumsum(irf_draws, 3);

    % Calculate cumulative IRFs for output (productivity + hours)
irf_output_orig = IRF_gali_orig(1,:,:) + IRF_gali_orig(2,:,:);
irf_output_draws = IRF_gali_draws(1,:,:,:) + IRF_gali_draws(2,:,:,:);

    % Combine all variables for plotting: [Prod, Hours, Output]
IRF_plot_orig = cumulate_irfs([IRF_gali_orig; irf_output_orig]);
IRF_plot_draws = cumulate_irfs([IRF_gali_draws; irf_output_draws]);

labels_plot_gali = ["Productivity", "Hours", "Output"];
plot_titles_gali = ["Technology Shock", "Non-Technology Shock"];
n_plot_vars = length(labels_plot_gali);
\end{lstlisting}

\begin{lstlisting}[language=Matlab]
% Plotting (Figure 2 Style)
IRF_plot_bands = prctile(IRF_plot_draws, [2.5, 97.5], 4);
fh = figure('Position',[50 50 980 760]);
tlo = tiledlayout(n_plot_vars, n_vars_gali, 'Padding', 'compact', 'TileSpacing', 'compact');

for i_var = 1:n_plot_vars
    for j_shock = 1:n_vars_gali
        nexttile; hold on; grid on;
        irf_point = squeeze(IRF_plot_orig(i_var, j_shock, :));
        irf_lo = squeeze(IRF_plot_bands(i_var, j_shock, :, 1));
        irf_hi = squeeze(IRF_plot_bands(i_var, j_shock, :, 2));
        fill([0:H_gali, H_gali:-1:0], [irf_lo', fliplr(irf_hi')], ...
            [0.85 0.9 1.0], 'EdgeColor','none', 'FaceAlpha', 0.7);
        plot(0:H_gali, irf_point, 'k-', 'LineWidth', 1.6);
        yline(0, 'k--');
        title(sprintf('Response of %s', labels_plot_gali(i_var)));
        ylabel('\% Deviation');
        if i_var == n_plot_vars, xlabel('Quarters'); end
        if i_var == 1, title(plot_titles_gali(j_shock)); end
    end
end

title(tlo, sprintf('Replication of Gali (1999) Figure 2 | VAR(%d), K=%d', p_gali, K_gali));
exportFig(fh, sprintf('ex2_gali1999_replication_VAR%d_K%d.pdf', p_gali, K_gali));
close(fh);
\end{lstlisting}

            \begin{figure}[H]
                \centering
                \includegraphics[width=0.94\linewidth]{output/ex2_gali1999_replication_VAR4_K1000.pdf}
                \caption{Estimated impulse responses from a bivariate model: U.S. data, first-differenced hours (point estimates and 2 standard error confidence intervals)}
                \label{fig:ex2_gali1999_replication_VAR4_K1000}
            \end{figure}

            Consistent with \cite{10.1257/aer.89.1.249}, a positive technology shock raises labor productivity on impact and persistently, while hours fall on impact (a short-lived but notable decline), and output rises more gradually. In contrast, a positive nontechnology shock pushes up hours and output persistently, with the effect on productivity fading to zero by construction of the identifying restriction. This pattern—negative conditional comovement of hours and productivity for technology shocks and positive conditional comovement for nontechnology shocks—matches Galí’s bivariate evidence. 
        
    \end{solution}

\printbibliography

\end{document}
