%Hey, if you're using this preamble it means that it was probably written by Stefano Graziosi (me). If you see something that doesn't make sense, feel free to email me at stefano.graziosi@studbocconi.it
%p.s. in case it's not already evident from the preamble, I'm not a professional LaTeX user, so I'm sure there are better ways to do things. I'm just trying to make it work.

% =============================================================================
%  Preamble maintained by Stefano Graziosi
%  If anything looks odd, ping: stefano.graziosi@studbocconi.it
%  LAST UPDATE: 01-11-2025
% =============================================================================

% I don't own copyright on anything; credits to original authors.

% =============================================================================
%  CORE PACKAGES (general utilities, fonts/encoding, colors, math, layout)
% =============================================================================

% Headers/footers
\usepackage{fancyhdr}

% Encoding & fonts
\usepackage[T1]{fontenc}
\usepackage{lmodern,mathrsfs}

\usepackage[dvipsnames]{xcolor}                     % Colors (named colors like MidnightBlue, Cerulean, etc.)
\usepackage[many]{tcolorbox}                        % Boxed environments
\usepackage{graphicx}                               % Graphics
\usepackage{bookmark}                               % (Optional, improves bookmark handling)
\usepackage{titlesec}                               % Section titles formatting
\usepackage{enumitem}                               % For fancy enumerates
\usepackage{listings}                               % For the coding environment

% Hyperlinks (kept in original order to preserve behavior)
\usepackage{hyperref}
\hypersetup{
  pdftitle={\jobname},
  pdfauthor={Stefano Graziosi},
  colorlinks=true,
  linkcolor=MidnightBlue,
  citecolor=ForestGreen,
  urlcolor=sgpurple
}

% Math
\usepackage{amsmath, amssymb, amsthm}
\usepackage{mathtools, amsfonts, bm}
\usepackage{thmtools}

% Line spacing & multicolumn text
\usepackage{setspace,multicol}

% Title image & PDF inclusion
\usepackage{titlepic}
\usepackage{pdfpages}

% Fun/extras
\usepackage{halloweenmath}
\usepackage{kantlipsum}

% =============================================================================
%  TABLES
% =============================================================================
\usepackage{array}
\usepackage{tabularx}
\usepackage{booktabs}
\usepackage{threeparttable}
\usepackage{siunitx}
\sisetup{
    round-mode=places,
    round-precision=3,
    table-number-alignment=center
}

\usepackage[nameinlink,noabbrev]{cleveref}

% =============================================================================
%  FIGURES & SUBFIGURES
% =============================================================================
% (graphicx loaded above intentionally to keep original load order)
\usepackage{wrapfig}
\usepackage{float}
\usepackage{caption}
\captionsetup{font=small, labelfont=bf, labelsep=period}
\usepackage{subcaption}

% =============================================================================
%  DIAGRAMS AND PLOTS
% =============================================================================

\usepackage[all]{xy}
\usepackage{tikz}
\usetikzlibrary{calc,arrows.meta,positioning}
% (Optional speed-up for heavy docs; requires -shell-escape)
% \usetikzlibrary{external}
% \tikzexternalize[prefix=tikzcache/]

% =============================================================================
%  Equations
% =============================================================================

\numberwithin{equation}{section} % or \counterwithin{equation}{section}

% =============================================================================
%  PAGE GEOMETRY
% =============================================================================
\usepackage[a4paper,margin=1in]{geometry}
% \usepackage[margin=1in]{geometry} % (alt)

% =============================================================================
%  BIBLIOGRAPHY
% =============================================================================
\usepackage[backend=biber,
            style=numeric,
            sorting=nyt]{biblatex}
\addbibresource{refs.bib}
\usepackage{csquotes}               % smarter quotes & works great with biblatex

% =============================================================================
%  CHAPTER/SECTION FORMATTING & NUMBERING
% =============================================================================

% Keep pure number for counters/refnames
\renewcommand\thesection{\arabic{section}}

% Pretty-print the section heading label
\titleformat{\section}{\normalfont\Large\bfseries}{Question \thesection:}{0.5em}{}

% Figure/Table numbering by section
\usepackage{chngcntr}
\counterwithin{figure}{section}
\counterwithin{table}{section}
\renewcommand{\thefigure}{\thesection-\arabic{figure}}
\renewcommand{\thetable}{\thesection-\arabic{table}}

% =============================================================================
%  COLORS (custom palette)
% =============================================================================

\definecolor{sgblue}{RGB}{0,169,211}
\definecolor{sggreen}{RGB}{0,164,0}
\definecolor{sgpurple}{RGB}{99,0,165}
\definecolor{sgyellow}{RGB}{255,211,0}
\definecolor{sgorange}{RGB}{255,127,20}
\definecolor{sbblue}{RGB}{219,248,254}
\definecolor{sbgreen}{RGB}{223,255,218}
\definecolor{sbpurple}{RGB}{241,220,255}

\definecolor{codegreen}{rgb}{0,0.6,0}
\definecolor{codegray}{rgb}{0.5,0.5,0.5}
\definecolor{codepurple}{rgb}{0.58,0,0.82}
\definecolor{backcolour}{rgb}{0.95,0.95,0.92}

% =============================================================================
%  GENERIC BOX ENVIRONMENT
% =============================================================================

% Standard LaTeX box with color accents
\newtcolorbox{mybox}[3][]{
  colframe = #2!25,
  colback  = #2!10,
  coltitle = #2!20!black,
  title    = {#3},
  #1,
}

% =============================================================================
%  BASIC PROBLEM/SOLUTION ENVIRONMENTS
% =============================================================================

% "Problem" environment
\newtheorem{problem}{Problem}

% Personalized "Solution" environment
\newenvironment{solution}[1][\it{\textcolor{MidnightBlue}{Solution}}]{\textbf{#1. } }{\textcolor{MidnightBlue}{$\square$}}

% =============================================================================
%  THEOREM STYLES & BOXED PRESENTATION
% =============================================================================

% A helper length to fine-tune the label alignment.
% We measure the width of a single normal-space character, store it in \spacelength,
% and later shift the heading by exactly that width (negative hskip) to line things up.
\newlength{\spacelength}
\settowidth{\spacelength}{\normalfont\ }

% ----------------------------- STYLE: "theorem" ------------------------------
% This defines a thmtools STYLE named "theorem".
% Later, any environment declared as \declaretheorem[style=theorem]{<env>}
% will use the fonts/formatting set below for its HEAD LINE (the bold label area).
\declaretheoremstyle[
  headfont={\bfseries\sffamily\footnotesize},  % font for the heading (e.g., "Theorem 2")
  notefont={\normalfont},                      % font for the optional note (the [...] part)
  bodyfont={\normalfont},                      % font for the body text inside the environment
  headpunct={\relax},                          % punctuation after the heading; \relax = none
  % headformat controls how the label is typeset.
  % \NAME   -> the environment name ("Theorem", "Proposition", etc.)
  % \NUMBER -> the auto number (e.g., "2.3")
  % \NOTE   -> the optional note, provided by \begin{theorem}[<NOTE>]
  % \marginparsep is the standard gap used for margin notes; you reuse it as a spacing constant.
  % \makebox[0pt][r]{...} puts the label in a zero-width box, right-aligned,
  % and then \hskip-\spacelength nudges the following text left by one space width
  % so that the body aligns nicely under the heading.
  headformat={\makebox[0pt][r]{\NAME\ \NUMBER\hspace{\marginparsep}}\hskip-\spacelength{\normalsize\NOTE}},
]{theorem}

% ----------------- tcolorbox skin for any env using name "theorem" -----------
% This does NOT create an environment; it says: "whenever the 'theorem'
% environment is used, wrap its contents with this box styling."
\tcolorboxenvironment{theorem}{
  boxrule=0pt,                                % no outer rectangular frame
  boxsep=0pt,                                 % no internal padding around the content (we set our own left/right)
  colback={White},                            % white background
  enhanced jigsaw,                            % enables advanced tcolorbox features
  borderline west={1pt}{0pt}{ForestGreen},    % a 1pt vertical colored bar on the LEFT
  sharp corners,                              % squared corners (not rounded)
  before skip=10pt,                           % vertical space before the box
  after skip=10pt,                            % vertical space after the box
  left=5pt, right=5pt,                        % inner left/right padding (content inset)
  breakable,                                  % allow the box to split across pages
}

% --------------------------- PROPOSITION setup -------------------------------
% Create "proposition" as a thmtools theorem that USES the "theorem" STYLE above.
\declaretheorem[style=theorem]{proposition}

% And give "proposition" its own box skin (purple left border). Same mechanics as "theorem".
\tcolorboxenvironment{proposition}{
  boxrule=0pt,
  boxsep=0pt,
  colback={White},
  enhanced jigsaw,
  borderline west={1pt}{0pt}{Mulberry},
  sharp corners,
  before skip=10pt,
  after skip=10pt,
  left=5pt,
  right=5pt,
  breakable,
}

% ---------------------------- THEOREM setup ----------------------------------
% Explicitly declare the "theorem" environment (thmtools way) using the "theorem" style.
% Note: the box style for "theorem" is already defined above via \tcolorboxenvironment{theorem}{...}.
\declaretheorem[style=theorem]{theorem}

% ----------------------------- STYLE: "proof" --------------------------------
% amsthm already defines a proof environment, so we first neutralize it:
\let\proof\relax
\let\endproof\relax

% Now we declare a thmtools STYLE named "proof" for the heading of the proof environment.
\declaretheoremstyle[
  headfont={\small\scshape},                  % small caps for "Proof"
  notefont={\normalfont},
  bodyfont={\normalfont},
  headpunct={\relax},
  headformat={\makebox[0pt][r]{\NAME\hspace{\marginparsep}}\hskip-\spacelength{\NOTE}},
]{proof}

% And we hook the "proof" environment to a tcolorbox with a thin black left rule.
\tcolorboxenvironment{proof}{
  boxrule=0pt,
  boxsep=0pt,
  blanker,                                    % removes the typical tcolorbox background to look like plain text
  borderline west={1pt}{0pt}{black},          % thin black bar on the left
  before skip=10pt,
  after skip=10pt,
  left=5pt,
  right=5pt,
  breakable,
}

% Now we REDECLARE the actual "proof" environment using the "proof" STYLE.
% The 'qed=\qedsymbol' option ensures the end-of-proof □ is inserted automatically at the end.
\declaretheorem[style=proof, qed=\qedsymbol]{proof}

% ------------------------------ STYLE: "claim" -------------------------------
% A lighter-looking style used for "Intuition" and similar notes (italic header).
\declaretheoremstyle[
  headfont={\footnotesize\itshape},
  notefont={\normalfont},
  bodyfont={\normalfont},
  headpunct={\relax},
  headformat={\makebox[0pt][r]{\NAME\hspace{\marginparsep}}\hskip-\spacelength{\NOTE}},
]{claim}

% Create an "Intuition" environment that uses the 'claim' style (no special box here).
\declaretheorem[style=claim]{Intuition}

% -------------------- Environments declared the amsthm way -------------------
% Below you switch to amsthm’s API: \theoremstyle{<name>} then \newtheorem{...}.
% NOTE: \theoremstyle selects an amsthm style (e.g., plain/definition/remar k),
% not a thmtools style. You happen to use the same style NAME "theorem",
% but that name is defined via thmtools above—not via amsthm’s \newtheoremstyle.
% In many setups, \theoremstyle{theorem} has no effect unless you also defined an
% amsthm style called "theorem". Your boxes still appear because
% \tcolorboxenvironment{<envname>}{...} wraps by ENVIRONMENT NAME.

\theoremstyle{theorem}             % (May be a no-op unless an amsthm style "theorem" exists.)
\newtheorem{ques}{Question}        % Unboxed label styling depends on current amsthm style.

% Definition with its own box skin (Cerulean left rule). The header font/styling here
% comes from the current amsthm \theoremstyle (see the note above).
\theoremstyle{theorem}
\newtheorem{definition}{Definition}
\tcolorboxenvironment{definition}{
  boxrule=0pt,
  boxsep=0pt,
  colback={White},
  enhanced jigsaw,
  borderline west={1pt}{0pt}{Cerulean},
  sharp corners,
  before skip=10pt,
  after skip=10pt,
  left=5pt,
  right=5pt,
  breakable,
}

% Lemma with a Rhodamine bar
\theoremstyle{theorem}
\newtheorem{lemma}{Lemma}
\tcolorboxenvironment{lemma}{
  boxrule=0pt,
  boxsep=0pt,
  blanker,
  borderline west={1pt}{0pt}{Rhodamine},
  before skip=10pt,
  after skip=10pt,
  sharp corners,
  left=5pt,
  right=5pt,
  breakable,
}

% Remark with a BurntOrange bar
\theoremstyle{theorem}
\newtheorem{remark}{Remark}
\tcolorboxenvironment{remark}{
  boxrule=0pt,
  boxsep=0pt,
  colback={White},
  enhanced jigsaw,
  borderline west={1pt}{0pt}{BurntOrange},
  before skip=10pt,
  after skip=10pt,
  sharp corners,
  left=5pt,
  right=5pt,
  breakable,
}

% Corollary with a WildStrawberry bar
\theoremstyle{theorem}
\newtheorem{corollary}{Corollary}
\tcolorboxenvironment{corollary}{
  boxrule=0pt,
  boxsep=0pt,
  enhanced jigsaw,
  borderline west={1pt}{0pt}{WildStrawberry},
  before skip=10pt,
  after skip=10pt,
  sharp corners,
  left=5pt,
  right=5pt,
  breakable,
}

% Example with a Dandelion bar
\theoremstyle{theorem}
\newtheorem{example}{Example}
\tcolorboxenvironment{example}{
  boxrule=0pt,
  boxsep=0pt,
  blanker,
  borderline west={1pt}{0pt}{Dandelion},
  before skip=10pt,
  after skip=10pt,
  sharp corners,
  left=5pt,
  right=5pt,
  breakable,
}

% Additional helpers using the 'claim' style (amsthm path).
% Again, \theoremstyle{claim} here refers to an amsthm style named "claim".
% You defined a thmtools style named "claim" above, which doesn't automatically
% register with amsthm. If no amsthm style "claim" exists, this line may be a no-op.
\theoremstyle{claim}
\newtheorem{intu}{Intuition}

\theoremstyle{claim}
\newtheorem{solu}{Solution}

% =============================================================================
%  CODE LISTINGS
% =============================================================================

\lstdefinestyle{mystyle}{
  backgroundcolor=\color{backcolour},
  commentstyle=\color{codegreen},
  keywordstyle=\color{magenta},
  numberstyle=\tiny\color{codegray},
  stringstyle=\color{codepurple},
  basicstyle=\ttfamily\footnotesize,
  breakatwhitespace=false,
  breaklines=true,
  captionpos=b,
  keepspaces=true,
  numbers=left,
  numbersep=5pt,
  showspaces=false,
  showstringspaces=false,
  showtabs=false,
  tabsize=2
}

\lstset{style=mystyle}
